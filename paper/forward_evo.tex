\subsection{ Forward Contracting with Embedded Volumetric Optionality}

We take a normal bill of lading and confer additional functionality so that the instrument itself would resemble a bearer forward contract with EVO capability. It is through the enhancement of the existing documents that we are able to deliver capacity in terms of physical movement of goods rather than just purely providing a cash-settlement for contract non-performance. 

\subsection{Example Market utilizing Forward EVO}

In our example we create a derivative off the underlying (principle) instrument, which is inventory warehoused amongst a few retailers.



\textbf{Provisions for Instrumentalization}

1. Visually replicate paper bills, preserving the often industry / custom specific layout;
2. Replicate the function of paper bills, including a party’s ability to issue, endorse, recut or surrender the e-bill;
3. Replicate the physical transfer of paper bills and, whilst all parties can view an e-bill \(in copy form\), the trading platform restricts access and endorsement to the current lawful holder \(by way of fob and access code\) as the e-bills securely move from one party to the next; and

4. Can be converted into paper bills, allowing them to be finally traded with parties which have not signed up to the ETS platform. However, it is notable that paper bills cannot be converted into e-bills since none of the ETS would be prepared to investigate and effectively warrant the provenance of paper bills and the physical cargo they represent.

 \textbf{Replacement of a transferable document or instrument with an electronic transferable record}

1. An electronic transferable record may replace a transferable document or instrument if a reliable method for the change of medium is used.

2. For the change of medium to take effect, a statement indicating a change of medium shall be inserted in the electronic transferable record.

3. Upon issuance of the electronic transferable record in accordance with paragraphs 1 and 2, the transferable document or instrument shall be made inoperative and ceases to have any effect or validity.

4. A change of medium in accordance with paragraphs 1 and 2 shall not affect the rights and obligations of the parties.
\footnote{\(Aug. 11, 1916, ch. 313, pt. C, § 11, as added Pub. L. 106–472, title II, § 201, Nov. 9, 2000, 114 Stat. 2065.\) \(a\) }


At the request of the depositor of an agricultural product stored or handled in a warehouse licensed under this chapter, the warehouse operator shall issue a receipt to the depositor as prescribed by the Secretary.

\(b\) **Actual storage required**

A receipt may not be issued under this section for an agricultural product unless the agricultural product is actually stored in the warehouse at the time of the issuance of the receipt.

\(c\) **Contents**

Each receipt issued for an agricultural product stored or handled in a warehouse licensed under this chapter shall contain such information, for each agricultural product covered by the receipt, as the Secretary may require by regulation.

\(d\) Prohibition on additional receipts or other documents

\(1\) Receipts

While a receipt issued under this chapter is outstanding and uncanceled by the warehouse operator, an additional receipt may not be issued for the same agricultural product \(or any portion of the same agricultural product\) represented by the outstanding receipt, except as authorized by the Secretary.

